\documentclass[11pt]{beamer}
\usetheme{Pittsburgh}
\usepackage[utf8]{inputenc}
\usepackage[spanish]{babel}
\usepackage{amsmath}
\usepackage{amsfonts}
\usepackage{amssymb}
\usepackage{graphicx}
\usepackage{caption}
\usepackage{subcaption}


\author{Luis Greco - Marcela Herrera - Cristian Kubrak - Alejo Salvador }
\title{Trabajo Práctico 3}
\subtitle{Tomografía computada}
%\setbeamercovered{transparent} 
%\setbeamertemplate{navigation symbols}{} 
%\logo{} 
%\institute{} 
%\date{} 
%\subject{} 
\begin{document}



\begin{frame}
\titlepage 
\end{frame}


\begin{frame}{INTRODUCCIÓN}
\begin{itemize}
\item El objetivo de este trabajo práctico es evaluar un método para reconstruir imágenes tomográficas sujetas a ruido, utilizando el método de aproximación por cuadrados mínimos.
\end{itemize}
\end{frame}

\begin{frame}{PREMISAS}
\begin{itemize}
\item Nuestro ''sujeto'' es una imagen de $n$x$n$ pixeles discretizada en celdas de $k$x$k$ pixeles.
\item La intensidad de un pixel se asocia al tiempo que demora un rayo en atravesar ese pixel.
\item La distancia que recorre un rayo que atraviesa al sujeto es igual a la cantidad de pixeles por los que pasa.
\item En cinemática $Velocidad = Espacio/Tiempo$. En este caso la ''velocidad'' promedio dentro de la celda $k$ es:
\begin{displaymath}
v_{k} =(\sum_{i=1}^{n}\sum_{j=1}^{n} I_{[i][j]})/n*n
\end{displaymath}
\item Los rayos están sujetos a ruido.
\end{itemize}
\end{frame}


\begin{frame}{DESARROLLO}
\frametitle {Discretización}
\begin{itemize}
\item Utilizamos valores divisores de la dimensión de la imagen (100 x 100 pixeles).
\item En los tests de granularidad, variamos el tamaño de las celdas tomando los valores 4x4, 5x5, 10x10, 20x20, 25x25 y 50x50 pixeles.
\item A medida que se achica el tamaño de las celdas crece el tiempo de procesamiento, lo cual condicionó la elección de los parámetros.
\end{itemize}
\end{frame}


\begin{frame}
\frametitle{Recorrido de un rayo}
\begin{itemize}
\item Los rayos son considerados como rectas en el plano, caracterizados por un punto de origen que llamamos $(x_{0},y_{0})$ y un ángulo $\alpha$.
\item Se busca la intersección con la recta $y=0$ la cual es el punto $(x-y/tangent(\alpha),0)$.
\item Usando la ecuación de la recta $y=tan(\alpha)*x+y_{0}$ y variando el valor de $x$, se calculan los puntos del plano por los que pasa el rayo.
\item Si el punto hallado pertenece a la imagen se guardan las coordenadas del pixel correspondiente en un vector.
\item Se usa el vector para el cálculo de distancias y de tiempos.
% agregar grafico
\end{itemize}
\end{frame}

\begin{frame}
\frametitle{Distancia recorrida por un rayo por celda}
\begin{itemize}
\item A partir del vector de pixeles por los que pasa un rayo, se averigua a qué celda pertenece. Se suma 1 unidad por pixel. El resultado se almacena en la matriz de distancias (una fila por rayos, una columna por celda numeradas de arriba hacia abajo y de izquierda a derecha).
\item Cada rayo toca a lo sumo $2n-1$ celdas. Se usa la estructura de matriz esparza del TP1.
\end{itemize}
\end{frame}


\begin{frame}
\frametitle{Recorrido de un rayo}
% TODO INCLUIR GRAFICO
\begin{figure}[H]
    \centering
    \includegraphics[scale=0.5]{img/recorridorayo.png}
    \caption{Celdas por la que pasa un rayo}
    \label{fig:recorridorayo}
\end{figure}
\end{frame}


\begin{frame}
\frametitle{Tiempo de recorrida de un rayo}
\begin{itemize}
\item A partir del vector de pixeles por los que pasa un rayo, se acumulan las intensidades de los mismos. Se almacenan los resultados en un vector de ''tiempos'' (un elemento por rayo).
\end{itemize}
\end{frame}

\begin{frame}
\frametitle{Sistema de ecuaciones}
% TODO INCLUIR MATRIZ DE ECUACIONES REALES Y DE CUADRADOS MINIMOS
% COMENTAR RELACION ENTER GRANULARIDAD, TAMAÑO DE LA MATRIZ Y CON EL TIEMPO DE PROCESAMIENTO
\end{frame}







\begin{frame}
\frametitle{Generación de rayos}
\framesubtitle{Cantidad y ubicación de los emisores}
% COMENTAR LAS ELECCIONES FALLIDAS. 
% SI SE PUEDE GRAFICARLAS
% DESCRIBIR SELECCION DE UBICACIONES Y CANTIDAD DE EMISORES Y DE RAYOS
% MENCIONAR LOS VALORES UTILIZADOS EN LA EXPERIMENTACION

\begin{enumerate}
    \item Primeras opciones:
    \begin{itemize}
        \item Trazar rayos horizontales y verticales, formando una cuadrícula $\Rightarrow$ Filas de ceros.
        \item Trazar rayos saliendo de los cuatro vértices de la imagen en distintos direcciones para barrer los $90^{\circ}$ de cada ángulo $\Rightarrow$ Información similar, imágenes irreconocibles.
    \end{itemize}
    \item Elección final:
    \begin{itemize}
        
    \item Trazar rayos desde los cuatro laterales de la imagen.
    \item Los emisores se ubican en igual cantidad sobre los lados de la imagen.
    \item La ubicación de los emisores se selecciona aleatoriamente con distribución uniforme entre $0$ y $k$ ($k$ cantidad de pixeles por lado de la imagen.
    \item Igual cantidad de rayos en ángulos que van entre $0^{\circ}$ a $180^{\circ}$.
    \item Ángulos seleccionados aleatoriamente con distribución uniforme.
    \end{itemize}

\end{enumerate}

\end{frame}


\begin{frame}
\frametitle{Generación de rayos}
\framesubtitle{Cantidad y ubicación de los emisores}
\begin{figure}[H] 
\centering
\includegraphics[scale=0.5]{img/rayos_tomo25x25px.png}
\caption{Trazado aleatorio de rayos}
\label{fig:rayos aleatorios}
\end{figure}
\end{frame}

\begin{frame}
    \frametitle{Generación de rayos}
    \framesubtitle{Casos fallidos}
    % incluir graficos e imagenes reconstruidas
    \begin{figure}[H]
        \centering
        \begin{subfigure}[h]{0.45\textwidth}
            \includegraphics[width=\textwidth]{img/rayosCuadricula.png}
            \caption{Cuadrícula}
            \label{fig:rayoscuadricula}
        \end{subfigure}%
        \hfill
        \begin{subfigure}[h]{0.45\textwidth}
                \includegraphics[width=\textwidth]{img/rayos_vertices.png}
                \caption{Desde los vértices}
                \label{fig:reconstruccion 10 px}
        \end{subfigure}
            
        \caption{Casos fallidos de distribución de emisores, sin ruido.}
    \end{figure}
\end{frame}



\begin{frame}
\frametitle{Ruido}
% DESCRIBIR COMO SE ELIGIO  PONERLO EN EL VECTOR INDEP
% DESCRIBIR ELECCION DEL CALCULO DEL RUIDO
% MENCIONAR VALORES UTILIZADOS EN LA EXPERIMENTACION
\end{frame}

%%%%%%%%%%%%%%%%%%%%%%%%%%%%%%%%%%%%%%%%%%%%%%%%%%%%%%%%%%%%%%%%%%%%%%%%%%%%%%%%%%%%%%%%%%%%%%%%%%%%

\begin{frame}{RESULTADOS}
    \frametitle{Imagen original}
    % incluir graficos e imagenes reconstruidas
    \begin{figure}
    \centering
            \includegraphics[scale=1]{img/tomo.png}
            \caption{Imagen original}
            \label{fig:original}
    \end{figure}
\end{frame}
    

\begin{frame}
\frametitle{ECM vs. tamaño de la celda}
% incluir graficos e imagenes reconstruidas
\begin{figure}
\centering
        \includegraphics[scale=0.4]{img/granu_ecm-eps-converted-to.pdf}
        \caption{Error cuadrático medio versus granularidad}
        \label{fig:ECM vs granularidad}
\end{figure}
\end{frame}


\begin{frame}
    \frametitle{ECM vs. tamaño de la celda}
    % incluir graficos e imagenes reconstruidas
    \begin{figure}[H]
    \centering
    \begin{subfigure}[h]{0.30\textwidth}
        \includegraphics[width=\textwidth]{img/tomo_granu_4.png}
        \caption{4x4px por celda}
        \label{fig:reconstruccion 4 px}
    \end{subfigure}%
    \hfill
    \begin{subfigure}[h]{0.30\textwidth}
            \includegraphics[width=\textwidth]{img/tomo_granu_10.png}
            \caption{4x4px por celda}
            \label{fig:reconstruccion 10 px}
        \end{subfigure}%
    \hfill
        \begin{subfigure}[h]{0.30\textwidth} 
            \includegraphics[width=\textwidth]{img/tomo_granu_20.png}
            \caption{10x10px por celda}
            \label{fig:reconstruccion 20 px}
        \end{subfigure}
        
        \caption{Reconstrucción variando granularidad, sin ruido}
    \end{figure}
\end{frame}
    
\begin{frame}
\frametitle{ECM vs. cantidad de emisores}

\begin{figure}[H]
    \centering
        
            \includegraphics[scale=0.4]{img/emi_ecm-eps-converted-to.pdf}
            \caption{ECM versus cantidad de emisores}
            \label{fig:ECM versus emisores}
        
    \end{figure}
\end{frame}


\begin{frame}
    \frametitle{ECM vs. cantidad de emisores}
    
    \begin{figure}[H]
        \centering
    
        \begin{subfigure}[h]{0.3\textwidth} 
            \includegraphics[width=\textwidth]{img/tomo_emisores_20.png}
            \caption{reconstrucción 20 emisores}
            \label{fig:reconstruccion 20 emisores}
        \end{subfigure}%
        \hfill
        \begin{subfigure}[h]{0.3\textwidth}
            \includegraphics[width=\textwidth]{img/tomo_emisores_60.png}
            \caption{reconstrucción 60 emisores}
            \label{fig:reconstruccion 60 emisores}
        \end{subfigure}%
        \hfill
        \begin{subfigure}[h]{0.3\textwidth} 
            \includegraphics[width=\textwidth]{img/tomo_emisores_140.png}
            \caption{reconstrucción 140 emisores}
            \label{fig:reconstruccion 140 emisores}
        \end{subfigure}
        
        \caption{Reconstrucción variando cantidad de emisores, sin ruido}
    \end{figure}
\end{frame}
    


\begin{frame}
    \frametitle{ECM vs. cantidad de rayos por emisor}
    
    \begin{figure}[H]
        \centering
            
                \includegraphics[scale=0.45]{img/cantrayos_ecm-eps-converted-to.pdf}
                \caption{ECM versus cantidad de rayos por emisor}
                \label{fig:ECM versus rayos por emisor}
            
        \end{figure}
\end{frame}
    


\begin{frame}
    \frametitle{ECM vs. cantidad de rayos por emisor}
    
    \begin{figure}[H]
        \centering
    
        \begin{subfigure}[h]{0.3\textwidth} 
            \includegraphics[width=\textwidth]{img/tomo_rayos_50.png}
            \caption{reconstrucción 50 rayos}
            \label{fig:reconstruccion 50 rayos}
        \end{subfigure}%
        \hfill
        \begin{subfigure}[h]{0.3\textwidth}
            \includegraphics[width=\textwidth]{img/tomo_rayos_100.png}
            \caption{reconstrucción 100 rayos}
            \label{fig:reconstruccion 100 rayos}
        \end{subfigure}%
        \hfill
        \begin{subfigure}[h]{0.3\textwidth} 
            \includegraphics[width=\textwidth]{img/tomo_rayos_140.png}
            \caption{reconstrucción 140 rayos}
            \label{fig:reconstruccion 140 rayos}
        \end{subfigure}
        \caption{Reconstrucción variando cantidad de rayos, sin ruido}
    \end{figure}
\end{frame}



\begin{frame}
\frametitle{Tiempo vs. granularidad}
\begin{figure}[H]
    \centering
        
            \includegraphics[scale=0.45]{img/granu_tiempo-eps-converted-to.pdf}
            \caption{Tiempo versus granularidad}
            \label{fig:tiempo versus granularidad}
        
    \end{figure}
\end{frame}


\begin{frame}
    \frametitle{Tiempo vs. cantidad de emisores}
    \begin{figure}[H]
        \centering
            
                \includegraphics[scale=0.45]{img/emi_tiempo-eps-converted-to.pdf}
                \caption{Tiempo versus cantidad de emisores}
                \label{fig:tiempo versus emisores}
            
    \end{figure}
\end{frame}
    

\begin{frame}
    \frametitle{Tiempo vs. cantidad de rayos por emisor}
    \begin{figure}[H]
        \centering
            
                \includegraphics[scale=0.45]{img/cantrayos_tiempo-eps-converted-to.pdf}
                \caption{Tiempo versus cantidad de rayos por emisor}
                \label{fig:tiempo versus cantrayos}
            
    \end{figure}
\end{frame}


\begin{frame}
\frametitle{Comparación de resultados}
% comparatica de imagenes con rayos de las esquinas y de la forma que %usanmos finalmente

\begin{figure}[H]
    \centering
    \begin{subfigure}[h]{0.3\textwidth} 
        \includegraphics[width=\textwidth]{img/tomovertices.png}
        \caption{Vértices}
        \label{fig:vertices}
    \end{subfigure}
    \begin{subfigure}[h]{0.3\textwidth} 
        \includegraphics[width=\textwidth]{img/tomovertices.png}
        \caption{Vértices2}
        \label{fig:vertices2}
    \end{subfigure}


\end{figure}
\end{frame}


\begin{frame}{CONCLUSIONES}
\end{frame}

\begin{frame}{REFERENCIAS}
\end{frame}


\begin{frame}{FIN}
\end{frame}



%\begin{frame}
%\tableofcontents
%\end{frame}


\end{document}

