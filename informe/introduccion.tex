\subsubsection*{Introducción}
\par El objetivo del trabajo práctico es evaluar un método para reconstruir imágenes tomográficas sujetas a ruido, utilizando el método de aproximación por cuadrados mínimos.

\par En la toma de una tomografía se emiten rayos X que atraviesan al sujeto en estudio y se realizan múltiples mediciones sobre los mismos para compensar los errores de medición que pudieran producirse. Este proceso conduce a un sistema de ecuaciones lineales sobredeterminado que en general no tiene solución, razón por la cual se utiliza el método de cuadrados mínimos para aproximar una solución.
En nuestro caso para simplificar el problema suponemos que las mediciones que se realizan son de los tiempos que les lleva a los rayos X atravesar el sujeto.

\par Los rayos se emiten de tal forma que atraviesan al sujeto en un plano dado y en distintos ángulos y con diferentes direcciones sobre ese plano.
\par En nuestro caso para simplificar el problema suponemos que las mediciones que se realizan son de los tiempos que les lleva a los rayos X atravesar el sujeto.
\par La superficie del corte a estudiar se discretiza dividiéndola en celdas de acuerdo a una cuadrícula de $n*n$, donde $n$ es un número entero. Cada celda puede ser atravesada por múltiples rayos, de hecho esto es recomendable para poder recolectar una cantidad de datos representativa que permita compensar errores numéricos.
Con los datos recolectados se plantea un sistema de ecuaciones lineales donde los coeficientes están dados por la distancia que recorre cada rayo dentro de cada una de las celdas de la cuadrícula, el vector independiente viene dado por los tiempos que demora de cada rayo en atravesar al sujeto y las incógnitas son las velocidades de los rayos dentro de cada celda. Estas velocidades son una propiedad de la materia atravesada.

\par Para evaluar la calidad de la reconstrucción de la imagen tomográfica utilizamos como métrica el cálculo del Error Cuadrático Medio, el cual va a comparar las velocidades de cada celda obtenidas a partir de las imágenes sin ruido contra las velocidades calculadas por medio de cuadrados mínimos.

\par Utilizamos la técnica de k-fold a la base de imágenes para generamos un conjunto de datos de entrenamiento y otro de datos de testing a fines de elegir los mejores valores para el $n$ de la discretización, el nivel de ruido y la cantidad y distribución de los rayos.

\par Centrándonos en nuestro caso, debido a que no es posible la utilización de un tomógrafo real para recolectar los datos, utilizamos imágenes de una base de datos de imágenes tomográficas (http://www.via.cornelledu/databases/). A partir de estas imágenes generamos las instancias de prueba que se van a convertir en nuestro "sujeto" en estudio tal como se detalla más abajo.

\par Usamos las imágenes de partida como si fuera el sujeto al que necesitamos estudiar y simulamos matemáticamente los rayos X, agregando ruido aleatorio para simular los errores de medición. Decidimos utilizar dos formas de manejar el ruido:
a) modificando la intensidad de los pixeles de la imagen elegida como sujeto, utilizando valores aleatorios con distribución gaussiana, para luego hacer la simulación de los rayos X sobre la imagen ruidosa.
b) Calculando los tiempos de recorrida de cada rayo simulado, a través de la imagen sin ruido, y luego perturbando los valores obtenidos.

\par La granularidad de la discretización, la cantidad y dirección de los rayos y el nivel de ruido utilizado son parámetros de experimentación. En su determinación tenemos en cuenta un compromiso entre la calidad de la imagen reconstruída y los tiempos de ejecución (principalmente en la resolución del sistema por cuadrados mínimos). Para la resolución del sistema utilizamos las ecuaciones normales y en este caso el aumento en la dimensión de la matriz generadas está íntimamente relacionada con el tiempo de procesamiento.

\subsubsection*{nombre}
