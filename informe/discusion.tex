Las expectativas que teníamos respecto de las pruebas usando solo KNN en comparación a KNN + PCA era que la segunda iba a dar mejores resultados en cuanto a las métricas de reconocimiento, sin embargo esto no fue así, lo que sí se logra usando PCA es trabajar con matrices más chicas, lo cual es útil si se trabaja con una base de datos con imágenes grandes o con muchas imágenes.
Con respecto a los tiempos tal como esperábamos PCA resulta lento en el procesamiento de la base de datos de entrenamiento sobre todo cuando se agregan muchas componentes principales. También suponíamos que las primeras componentes principales iban a influir más en tener buenos resultados de reconocimiento. Esto efectivamente fue así y lo usamos al diseñar los casos de test usando $\alpha$ más próximos en los valores pequeños y espaciándolos en valores más altos.
Una de las cosas que suponíamos es que usando un K más alto en KNN iba a funcionar mejor, pero esto no fue así, dando mejores métricas para K más chicos.
Los tiempos de ejecución fue una de las cuestiones que tuvimos en cuenta a a hora de diseñar los casos de pruebafundamentalmente en las corridas que usan PCA.