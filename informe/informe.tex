%\documentclass[a4paper]{article}
\documentclass[10pt,a4paper]{article}
\usepackage[utf8]{inputenc} % para poder usar tildes en archivos UTF-8
\usepackage[spanish,es-tabla]{babel}
\usepackage{verbatim}
\usepackage{clrscode3e}
\usepackage{amssymb}
\usepackage{graphicx}
\usepackage{float}
\usepackage{pdfpages}

% \usepackage{bibtex}

%\usepackage{a4wide} % márgenes un poco más anchos que lo usual

\usepackage{caratula} % Se puede descargar en ~> https://github.com/bcardiff/dc-tex
\usepackage[breaklinks=true]{hyperref}


\begin{document} % Todo lo que escribamos a partir de aca va a aparecer en el documento.

%fran
%\sloppy

% Completar los datos de la caratula
\titulo{Trabajo Práctico 2 - Reconocimiento de Imagenes} 
\fecha{\today}
\materia{Métodos Numéricos}
\grupo{Grupo "Nombre Del Grupo"}

% Completar los integrantes del grupo:)
%\integrante{Facundo, Araujo}{321/15}{facalj\_velez@hotmail.com}
\integrante{Cristian, Kubrak}{456/15}{Kubrakcristian@gmail.com}
\integrante{Marcela Alejandra, Herrera}{1162/84}{marcelaalejandraherrera@yahoo.com.ar}
\integrante{Luis Fernando, Greco}{150/15}{luifergreco@gmail.com}


\maketitle

% Aca comienzan a escribir su informe
\tableofcontents

\newpage

\section{Introducción}
\subsubsection*{Introducción}
\par El objetivo del trabajo práctico es evaluar un método para reconstruir imágenes tomográficas sujetas a ruido, utilizando el método de aproximación por cuadrados mínimos.

\par En la toma de una tomografía se emiten rayos X que atraviesan al sujeto en estudio y se realizan múltiples mediciones sobre los mismos para compensar los errores de medición que pudieran producirse. Este proceso conduce a un sistema de ecuaciones lineales sobredeterminado que en general no tiene solución, razón por la cual se utiliza el método de cuadrados mínimos para aproximar una solución.
En nuestro caso para simplificar el problema suponemos que las mediciones que se realizan son de los tiempos que les lleva a los rayos X atravesar el sujeto.

\par Los rayos se emiten de tal forma que atraviesan al sujeto en un plano dado y en distintos ángulos y con diferentes direcciones sobre ese plano.
\par En nuestro caso para simplificar el problema suponemos que las mediciones que se realizan son de los tiempos que les lleva a los rayos X atravesar el sujeto.
\par La superficie del corte a estudiar se discretiza dividiéndola en celdas de acuerdo a una cuadrícula de $n*n$, donde $n$ es un número entero. Cada celda puede ser atravesada por múltiples rayos, de hecho esto es recomendable para poder recolectar una cantidad de datos representativa que permita compensar errores numéricos.
Con los datos recolectados se plantea un sistema de ecuaciones lineales donde los coeficientes están dados por la distancia que recorre cada rayo dentro de cada una de las celdas de la cuadrícula, el vector independiente viene dado por los tiempos que demora de cada rayo en atravesar al sujeto y las incógnitas son las velocidades de los rayos dentro de cada celda. Estas velocidades son una propiedad de la materia atravesada.

\par Para evaluar la calidad de la reconstrucción de la imagen tomográfica utilizamos como métrica el cálculo del Error Cuadrático Medio, el cual va a comparar las velocidades de cada celda obtenidas a partir de las imágenes sin ruido contra las velocidades calculadas por medio de cuadrados mínimos.

\par Utilizamos la técnica de k-fold a la base de imágenes para generamos un conjunto de datos de entrenamiento y otro de datos de testing a fines de elegir los mejores valores para el $n$ de la discretización, el nivel de ruido y la cantidad y distribución de los rayos.

\par Centrándonos en nuestro caso, debido a que no es posible la utilización de un tomógrafo real para recolectar los datos, utilizamos imágenes de una base de datos de imágenes tomográficas (http://www.via.cornelledu/databases/). A partir de estas imágenes generamos las instancias de prueba que se van a convertir en nuestro "sujeto" en estudio tal como se detalla más abajo.

\par Usamos las imágenes de partida como si fuera el sujeto al que necesitamos estudiar y simulamos matemáticamente los rayos X, agregando ruido aleatorio para simular los errores de medición. Decidimos utilizar dos formas de manejar el ruido:
a) modificando la intensidad de los pixeles de la imagen elegida como sujeto, utilizando valores aleatorios con distribución gaussiana, para luego hacer la simulación de los rayos X sobre la imagen ruidosa.
b) Calculando los tiempos de recorrida de cada rayo simulado, a través de la imagen sin ruido, y luego perturbando los valores obtenidos.

\par La granularidad de la discretización, la cantidad y dirección de los rayos y el nivel de ruido utilizado son parámetros de experimentación. En su determinación tenemos en cuenta un compromiso entre la calidad de la imagen reconstruída y los tiempos de ejecución (principalmente en la resolución del sistema por cuadrados mínimos). Para la resolución del sistema utilizamos las ecuaciones normales y en este caso el aumento en la dimensión de la matriz generadas está íntimamente relacionada con el tiempo de procesamiento.

\subsubsection*{nombre}

\newpage

% \section{Demostraciones}
% \newpage

\section{Desarrollo}
\subsubsection*{Desarrollo}

\title{Generación de rayos}

\par Una de las decisiones que tuvimos que tomar fue de qué forma trazar los rayos simulados que íbamos a aplicar a la imagen.

\par En cuanto a la cantidad de rayos, la misma debe ser mayor que $n^{2}$, siendo $n$ la cantidad de celdas de la discretización, ya que con una cantidad menor habría celdas que no son atravesadas por ningún rayo. En ese caso, hay columnas completas de ceros en la matriz de distancias, lo cual se traduce en filas de ceros en la matriz del sistema a resolver que como ya explicamos en la introducción se obtiene como $D^{t}D t = D^{t} v$. Esta situación se corresponde con un sistema con infinitas soluciones.

\par En cuanto a la ubicación de los emisores, inicialmente pensamos en dos opciones que resultaron fallidas:
\begin{enumerate}
\item Trazar rayos horizontales y verticales, formando una cuadrícula.
\item Trazar rayos saliendo de los cuatro vértices de la imagen en distintos direcciones para barrer los 90^{°} de cada ángulo.
\end{enumerate}

\par En ambos casos, al comenzar la experimentación nos encontramos con dificultades.

\par En el primer caso al realizar la eliminación gaussiana nos encontrábamos con que la matriz tenía filas completas en cero (matriz no inversible).

\par En el segundo caso, si bien se ejecutaban todos los pasos del programa las imágenes reconstruídas no eran reconocibles.

Al realizar un análisis de estos resultados nos dimos cuenta que los rayos trazados de estas formas eran muy similares entre sí y en consecuencia no suministraban información sufuciente para realizar una buena aproximación a la verdadera solución del sistema. Igualmente adjuntamos la implementación en el código a título informativo.

\par Finalmente decidimos trazar rayos desde los cuatro laterales de la imagen. Los emisores se ubican en igual cantidad sobre los lados de la imagen, la ubicación de los emisores se selecciona aleatoriamente en base a una distribución uniforme entre $0$ y $k$, siendo $k$ la cantidad de pixeles por lado de la imagen (es decir suponiendo que la imagen tiene $k*k$ pixeles).

\par De cada uno de estos emisores salen igual cantidad de rayos en ángulos que van entre $0^{\circ}$ a $180^{\circ}$. Los valores de los ángulos también se seleccionan aleatoriamente con distribución uniforme.

\par Si bien este método es aleatorio -con lo que los resultados difícilmente se repitan en corridas sucesivas- utilizando una cantidad suficientemente grande de rayos (como en este caso) por consecuencia de la Ley de los Grandes Números, en general los rayos quedan bien distribuídos a lo largo (y ancho) de la imagen, pudiendo obtener así resultados consistentes en varias corridas.

\par De los tres métodos este fue el que nos dio mejores resultados y el que terminamos eligiendo para realizar la experimentación.

\par Durante la experimentación, para una discretización de imagen de 10 celdas por lado (100 celdas en total), realizamos dos tipos de pruebas:
\begin{enumerate}
\item Sobre la cantidad de emisores, variando entre 20 y 140, aumentando de a 20 ($2*n$ hasta $7*2*n$). Sin aplicar ruido y manteniendo constante la cantidad de rayos por emisor en 100.
\item Sobre la cantidad de rayos por emisor, variando desde 50 hasta 150, aumentando de a 10.  Sin aplicar ruido y manteniendo constante la cantidad de emisores en 100.
\end{enumerate}

\newpage

\section{Experimentación}
\subsubsection*{Experimentación}
Para experimentar analizamos la influencia en los resultados de las diferentes variables de experimentación que manejamos utilizando tanto el método KNN como el KNN + PCA. 
Las mismas son la cantidad de vecinos considerados por KNN (k), y la cantidad de componentes principales de la imagen transformada ($\alpha$).
Lo hicimos sobre dos bases de datos: una con im\'agenes de tamaño reducido, y otra con imágenes sin reducir (a las cuales llamaremos \textit{big tempo})
Nuestras expectativas previas a la experimentación fueron las siguientes\\
Por un lado consideramos que el tamaño de las im\'agenes influir\'a en el tiempo requerido para su procesamiento pero debido a la mayor informaci\'on disponible,
funciar\'an mejores los algoritmos.\\
Con respecto a la variaci\'on del $k$ en $KNN$ suponemos que con un $k$ ni muy grande ni muy chico, obtendremos buenos resulatados de reconocimiento.\\
Frente a los diferentes algoritmos (KNN o PCA + KNN) creemos que la segunda opci\'on realizar\'a un mejor trabajo pero en un mayor tiempo, por lo que habr\'a que evaluar si se justifica su utilizanci\'on.\\
Por \'ultimo, respecto al par\'ametro $\alpha$ (cantidad de iteraciones del m\'etodo de las potencias) resulta obvio estimar que a mayor $\alpha$, mayor ser\'a el tiempo de ejecuc\'on aunque esperamos que esto
mejore significativamente la tasa de reconocimiento.




% Conclusión:
% Luego de observar estos gráficos llegamos a algunas conclusiones.
% En cuanto al tiempo, por un lado, la cantidad de vecinos cercanos que tomemos no afecta significativamente el tiempo, pero lo que sí lo afecta es el $\alpha$ de PCA.
% A qué se debe esto? Teniendo en cuenta el funcionamiento de nuestro algoritmo, entendemos que esto se debe a que una gran parte del tiempo de procesamiento de PCA se consume en el método de la potencia (que se realiza $\alpha$ veces) y en la transformación de los autovectores calculados en los de la verdadera matriz de covarianza de la muestra, que involucran numerosos cálculos matriciales.

% Sin embargo, pensamos que en una implementación real estaríamos trabajando con una única training base, y las transformaciones que llevamos a cabo en el PCA las haríamos una única vez, con lo que este costo de tiempo se pagaría solamente una vez o cuando sea necesario agregar o quitar alguna imagen, para luego realizar únicamente el reconocimiento. Usando PCA + KNN tenemos la ventaja de trabajar con imágenes de menor tamaño con el consiguiente ahorro de espacio.
\newpage

\section{Resultados}
\subsubsection*{Resultados obtenidos}

\begin{figure}[H]
	\centering	\includegraphics[width=0.8\textwidth]{img/nombre.png}
	\caption{Titulo}
	\label{fig:etiqueta}
\end{figure}


\newpage


\section{Discusión}
\subsubsection*{Discusion}

\par En cuanto al ruido, para verificar lo expicado anteriormente acerca de nuestra elección de ruido multiplicativo hicimos los siguientes experimentos.
Utilizamos una imagen de 100px x 100px, con $\alpha$ = 0.2, dividida en 5 celdas y con 100 emisores de rayos, y 100 rayos emitidos desde cada uno de estos. Luego, sobre esto comparamos el vector de tiempos de los rayos sin ruido contra el mismo con ruido multiplicativo y aditivo. 
\newpage

\section{Conclusiones}
\subsubsection*{Conclusiones}


\par El método utilizado para la generación de rayos permitió realizar una reconstrucción aceptable de las imágenes, no obstante, nos parece pertinente mencionar que si llegásemos a necesitar una implementación real de un tomógrafo la elección aleatoria de las ubicaciones de emisores y de los ángulos podría no ser la mejor. En este sentido sería superador a futuro encarar otra implementación que permita parámetros no aleatorios.

\par En este caso encontramos un buen comportamiento de nuestro sistema en relación al ruido gaussiano, nos parece interesante a futuro pensar cómo puede verse afectado por otros tipos de ruido tal vez menos comunes pero que igualmente pueden presentarse en algunas ocasiones. 

\par La cantidad de rayos emitidos no evidenció ser un factor determinante en la calidad de la imagen reconstruída dada la forma de trazarlos usada en este trabajo ya que mientras que el método se complete la calidad de la imagen es alta en la mayoría de los casos. En cambio la granularidad elegida juega un papel clave ya que a menor tamaño de celda los cálculos permiten una mejor aproximación de los valores de intensidad a calcular. Tomando un tamaño de celda mayor se termina aproximando por un valor ''promedio'' de los valores de los pixeles de la celda con lo cual se pierde información particular de los mismos.

\par En cambio la localizacion de los emisores si parecio tener una influencioa importante en la mejora de los resultados respecto de los primeros intentos fallidos que realizamos usando la grilla y los rayos trazados desde los vértices.

\par Tampoco pudimos establecer una relación directa entre el error cuadrático medio y la resolución final de la imagen observada a simple vista. No nos pareció una métrica que funcionara de manera muy apropiada en este caso.

\par Con respecto a los tiempos de procesamiento lo que más influye en los mismos es el tiempo de resolución del sistema de ecuaciones de cuadrados mínimos el cual crece mucho de tamaño al disminuir la granularidad (como se explicó en mayor detalle en la Introducción).
Esto condiciono la eleccion de las imagenes y de la granularidad.




\newpage

\section{Apendices}
\subsubsection*{Apendices}

\textbf{}Parámetros de entrada:} Para correr el programa se deben utilizar los siguientes parámetros:
\begin{enumerate}
\item Nombre del archivo de entrada sin extensión. Se asume que el archivo es de tipo csv conteniendo imágenes de 16 bits.
\item Nivel de ruido: valor entre 0 y 1.
\item Dimension de la imagen: cantidad de pixeles por fila. Asumimos que la imagen es cuadrada.
\item Dimensión de celda. Número entero que debe ser divisor de la cantidad de pixeles por fila de la imagen. Se asume que las celdas van a ser cuadradas.
\item Cantidad de Emisores. Cantidad total de emisores.
\item Cantidad de rayos. Cantidad de rayos que se van a trazar desde cada emisor.
\end{enumerate}
\par Ejemplo: ./tp3 ./imgs_TC/tomo 0.1 100 5 20 120
\par Con estos parámetros se toma el archivo tomo.csv, nivel de ruido 0.1, dimension de la imagen 100x100 pixeles, dimensión de la celda 5x5 pixeles, 20 emisores, 120 rayos por emisor.

\par En el mismo directorio en que se encuentran las imágenes de entrada se genera el archivo que contiene la imagen reconstruida en format pgm de 8 bits. Tiene el mismo nombre que el archivo de entrada y extensión '.pgm'.


	
\newpage

\end{document}

